\documentclass[]{article}
\usepackage{lmodern}
\usepackage{amssymb,amsmath}
\usepackage{ifxetex,ifluatex}
\usepackage{fixltx2e} % provides \textsubscript
\ifnum 0\ifxetex 1\fi\ifluatex 1\fi=0 % if pdftex
  \usepackage[T1]{fontenc}
  \usepackage[utf8]{inputenc}
\else % if luatex or xelatex
  \ifxetex
    \usepackage{mathspec}
  \else
    \usepackage{fontspec}
  \fi
  \defaultfontfeatures{Ligatures=TeX,Scale=MatchLowercase}
\fi
% use upquote if available, for straight quotes in verbatim environments
\IfFileExists{upquote.sty}{\usepackage{upquote}}{}
% use microtype if available
\IfFileExists{microtype.sty}{%
\usepackage{microtype}
\UseMicrotypeSet[protrusion]{basicmath} % disable protrusion for tt fonts
}{}
\usepackage[margin=1in]{geometry}
\usepackage{hyperref}
\hypersetup{unicode=true,
            pdftitle={Ejercicio 4},
            pdfauthor={Abraham Nieto 51556},
            pdfborder={0 0 0},
            breaklinks=true}
\urlstyle{same}  % don't use monospace font for urls
\usepackage{color}
\usepackage{fancyvrb}
\newcommand{\VerbBar}{|}
\newcommand{\VERB}{\Verb[commandchars=\\\{\}]}
\DefineVerbatimEnvironment{Highlighting}{Verbatim}{commandchars=\\\{\}}
% Add ',fontsize=\small' for more characters per line
\usepackage{framed}
\definecolor{shadecolor}{RGB}{248,248,248}
\newenvironment{Shaded}{\begin{snugshade}}{\end{snugshade}}
\newcommand{\KeywordTok}[1]{\textcolor[rgb]{0.13,0.29,0.53}{\textbf{#1}}}
\newcommand{\DataTypeTok}[1]{\textcolor[rgb]{0.13,0.29,0.53}{#1}}
\newcommand{\DecValTok}[1]{\textcolor[rgb]{0.00,0.00,0.81}{#1}}
\newcommand{\BaseNTok}[1]{\textcolor[rgb]{0.00,0.00,0.81}{#1}}
\newcommand{\FloatTok}[1]{\textcolor[rgb]{0.00,0.00,0.81}{#1}}
\newcommand{\ConstantTok}[1]{\textcolor[rgb]{0.00,0.00,0.00}{#1}}
\newcommand{\CharTok}[1]{\textcolor[rgb]{0.31,0.60,0.02}{#1}}
\newcommand{\SpecialCharTok}[1]{\textcolor[rgb]{0.00,0.00,0.00}{#1}}
\newcommand{\StringTok}[1]{\textcolor[rgb]{0.31,0.60,0.02}{#1}}
\newcommand{\VerbatimStringTok}[1]{\textcolor[rgb]{0.31,0.60,0.02}{#1}}
\newcommand{\SpecialStringTok}[1]{\textcolor[rgb]{0.31,0.60,0.02}{#1}}
\newcommand{\ImportTok}[1]{#1}
\newcommand{\CommentTok}[1]{\textcolor[rgb]{0.56,0.35,0.01}{\textit{#1}}}
\newcommand{\DocumentationTok}[1]{\textcolor[rgb]{0.56,0.35,0.01}{\textbf{\textit{#1}}}}
\newcommand{\AnnotationTok}[1]{\textcolor[rgb]{0.56,0.35,0.01}{\textbf{\textit{#1}}}}
\newcommand{\CommentVarTok}[1]{\textcolor[rgb]{0.56,0.35,0.01}{\textbf{\textit{#1}}}}
\newcommand{\OtherTok}[1]{\textcolor[rgb]{0.56,0.35,0.01}{#1}}
\newcommand{\FunctionTok}[1]{\textcolor[rgb]{0.00,0.00,0.00}{#1}}
\newcommand{\VariableTok}[1]{\textcolor[rgb]{0.00,0.00,0.00}{#1}}
\newcommand{\ControlFlowTok}[1]{\textcolor[rgb]{0.13,0.29,0.53}{\textbf{#1}}}
\newcommand{\OperatorTok}[1]{\textcolor[rgb]{0.81,0.36,0.00}{\textbf{#1}}}
\newcommand{\BuiltInTok}[1]{#1}
\newcommand{\ExtensionTok}[1]{#1}
\newcommand{\PreprocessorTok}[1]{\textcolor[rgb]{0.56,0.35,0.01}{\textit{#1}}}
\newcommand{\AttributeTok}[1]{\textcolor[rgb]{0.77,0.63,0.00}{#1}}
\newcommand{\RegionMarkerTok}[1]{#1}
\newcommand{\InformationTok}[1]{\textcolor[rgb]{0.56,0.35,0.01}{\textbf{\textit{#1}}}}
\newcommand{\WarningTok}[1]{\textcolor[rgb]{0.56,0.35,0.01}{\textbf{\textit{#1}}}}
\newcommand{\AlertTok}[1]{\textcolor[rgb]{0.94,0.16,0.16}{#1}}
\newcommand{\ErrorTok}[1]{\textcolor[rgb]{0.64,0.00,0.00}{\textbf{#1}}}
\newcommand{\NormalTok}[1]{#1}
\usepackage{graphicx,grffile}
\makeatletter
\def\maxwidth{\ifdim\Gin@nat@width>\linewidth\linewidth\else\Gin@nat@width\fi}
\def\maxheight{\ifdim\Gin@nat@height>\textheight\textheight\else\Gin@nat@height\fi}
\makeatother
% Scale images if necessary, so that they will not overflow the page
% margins by default, and it is still possible to overwrite the defaults
% using explicit options in \includegraphics[width, height, ...]{}
\setkeys{Gin}{width=\maxwidth,height=\maxheight,keepaspectratio}
\IfFileExists{parskip.sty}{%
\usepackage{parskip}
}{% else
\setlength{\parindent}{0pt}
\setlength{\parskip}{6pt plus 2pt minus 1pt}
}
\setlength{\emergencystretch}{3em}  % prevent overfull lines
\providecommand{\tightlist}{%
  \setlength{\itemsep}{0pt}\setlength{\parskip}{0pt}}
\setcounter{secnumdepth}{0}
% Redefines (sub)paragraphs to behave more like sections
\ifx\paragraph\undefined\else
\let\oldparagraph\paragraph
\renewcommand{\paragraph}[1]{\oldparagraph{#1}\mbox{}}
\fi
\ifx\subparagraph\undefined\else
\let\oldsubparagraph\subparagraph
\renewcommand{\subparagraph}[1]{\oldsubparagraph{#1}\mbox{}}
\fi

%%% Use protect on footnotes to avoid problems with footnotes in titles
\let\rmarkdownfootnote\footnote%
\def\footnote{\protect\rmarkdownfootnote}

%%% Change title format to be more compact
\usepackage{titling}

% Create subtitle command for use in maketitle
\newcommand{\subtitle}[1]{
  \posttitle{
    \begin{center}\large#1\end{center}
    }
}

\setlength{\droptitle}{-2em}

  \title{Ejercicio 4}
    \pretitle{\vspace{\droptitle}\centering\huge}
  \posttitle{\par}
    \author{Abraham Nieto 51556}
    \preauthor{\centering\large\emph}
  \postauthor{\par}
      \predate{\centering\large\emph}
  \postdate{\par}
    \date{26 de septiembre de 2018}


\begin{document}
\maketitle

EJERCICIO 4. Un investigador desea evaluar la relación entre el salario
anual de trabajadores de una compañía de nivel medio y alto (Y, en miles
de dólares) y el índice de calidad de trabajo (X1), número de años de
experiencia (X2) y el índice de éxito en publicaciones (X3). La muestra
consiste de 24 trabajadores. Realiza un análisis Bayesiano completo de
los datos y obtén las predicciones de salarios para 3 nuevos empleados
con variables explicativas: x1F `   5.4,17,6.0 , x' 6.2,12,5.8 2F 
y x ' 6.4,21,6.1 1F  .

\begin{Shaded}
\begin{Highlighting}[]
\KeywordTok{library}\NormalTok{(R2OpenBUGS)}
\KeywordTok{library}\NormalTok{(R2jags)}
\end{Highlighting}
\end{Shaded}

\begin{verbatim}
## Loading required package: rjags
\end{verbatim}

\begin{verbatim}
## Loading required package: coda
\end{verbatim}

\begin{verbatim}
## Linked to JAGS 4.2.0
\end{verbatim}

\begin{verbatim}
## Loaded modules: basemod,bugs
\end{verbatim}

\begin{verbatim}
## 
## Attaching package: 'R2jags'
\end{verbatim}

\begin{verbatim}
## The following object is masked from 'package:coda':
## 
##     traceplot
\end{verbatim}

\begin{Shaded}
\begin{Highlighting}[]
\NormalTok{wdir<-}\StringTok{"/home/abraham/RA2018/tareas"}
\KeywordTok{setwd}\NormalTok{(wdir)}
\end{Highlighting}
\end{Shaded}

\begin{Shaded}
\begin{Highlighting}[]
\CommentTok{#--- Funciones utiles ---}
\NormalTok{prob<-}\ControlFlowTok{function}\NormalTok{(x)\{}
\NormalTok{  out<-}\KeywordTok{min}\NormalTok{(}\KeywordTok{length}\NormalTok{(x[x}\OperatorTok{>}\DecValTok{0}\NormalTok{])}\OperatorTok{/}\KeywordTok{length}\NormalTok{(x),}\KeywordTok{length}\NormalTok{(x[x}\OperatorTok{<}\DecValTok{0}\NormalTok{])}\OperatorTok{/}\KeywordTok{length}\NormalTok{(x))}
\NormalTok{  out}
\NormalTok{\}}
\end{Highlighting}
\end{Shaded}

\begin{Shaded}
\begin{Highlighting}[]
\NormalTok{sal<-}\KeywordTok{read.table}\NormalTok{(}\StringTok{"http://allman.rhon.itam.mx/~lnieto/index_archivos/salarios.txt"}\NormalTok{,}\DataTypeTok{header=}\OtherTok{TRUE}\NormalTok{)}
\KeywordTok{head}\NormalTok{(sal)}
\end{Highlighting}
\end{Shaded}

\begin{verbatim}
##      Y  X1 X2  X3
## 1 33.2 3.5  9 6.1
## 2 40.3 5.3 20 6.4
## 3 38.7 5.1 18 7.4
## 4 46.8 5.8 33 6.7
## 5 41.4 4.2 31 7.5
## 6 37.5 6.0 13 5.9
\end{verbatim}

\begin{Shaded}
\begin{Highlighting}[]
\NormalTok{n<-}\KeywordTok{nrow}\NormalTok{(sal)}
\NormalTok{x1f<-}\KeywordTok{c}\NormalTok{(}\FloatTok{5.4}\NormalTok{,}\DecValTok{17}\NormalTok{,}\FloatTok{6.0}\NormalTok{)}
\NormalTok{x2f<-}\KeywordTok{c}\NormalTok{(}\FloatTok{6.2}\NormalTok{,}\DecValTok{12}\NormalTok{,}\FloatTok{5.8}\NormalTok{)}
\NormalTok{x3f<-}\KeywordTok{c}\NormalTok{(}\FloatTok{6.4}\NormalTok{,}\DecValTok{21}\NormalTok{,}\FloatTok{6.1}\NormalTok{)}

\NormalTok{m<-}\KeywordTok{length}\NormalTok{(x1f)}
\CommentTok{#-Defining data-}
\CommentTok{#data<-list(x1=sal$X1,x2=sal$X2,x3=sal$X3)}
\NormalTok{data<-}\KeywordTok{list}\NormalTok{(}\StringTok{"n"}\NormalTok{=n,}\StringTok{"y"}\NormalTok{=sal}\OperatorTok{$}\NormalTok{Y,}\StringTok{"x1"}\NormalTok{=sal}\OperatorTok{$}\NormalTok{X1,}\StringTok{"x2"}\NormalTok{=sal}\OperatorTok{$}\NormalTok{X2,}\StringTok{"x3"}\NormalTok{=sal}\OperatorTok{$}\NormalTok{X3,}\StringTok{"m"}\NormalTok{=m,}\StringTok{"x1f"}\NormalTok{=x1f,}\StringTok{"x2f"}\NormalTok{=x2f,}\StringTok{"x3f"}\NormalTok{=x3f)}
\CommentTok{#-Defining inits-}
\NormalTok{inits<-}\ControlFlowTok{function}\NormalTok{()\{}\KeywordTok{list}\NormalTok{(}\DataTypeTok{beta=}\KeywordTok{rep}\NormalTok{(}\DecValTok{0}\NormalTok{,}\DecValTok{4}\NormalTok{),}\DataTypeTok{tau=}\DecValTok{1}\NormalTok{,}\DataTypeTok{yf=}\KeywordTok{rep}\NormalTok{(}\DecValTok{0}\NormalTok{,n),}\DataTypeTok{yf1=}\KeywordTok{rep}\NormalTok{(}\DecValTok{0}\NormalTok{,}\DecValTok{3}\NormalTok{))\} }
\CommentTok{#-Selecting parameters to monitor-}
\NormalTok{parameters<-}\KeywordTok{c}\NormalTok{(}\StringTok{"beta"}\NormalTok{,}\StringTok{"tau"}\NormalTok{,}\StringTok{"sig2"}\NormalTok{,}\StringTok{"yf"}\NormalTok{,}\StringTok{"yf1"}\NormalTok{)}
\CommentTok{#OpenBUGS}
\NormalTok{ej4.sim<-}\KeywordTok{bugs}\NormalTok{(data,inits,parameters,}\DataTypeTok{model.file=}\StringTok{"Ej4t.txt"}\NormalTok{,}
              \DataTypeTok{n.iter=}\DecValTok{5000}\NormalTok{,}\DataTypeTok{n.chains=}\DecValTok{1}\NormalTok{,}\DataTypeTok{n.burnin=}\DecValTok{500}\NormalTok{,}\DataTypeTok{n.thin =} \DecValTok{1}\NormalTok{)}
\end{Highlighting}
\end{Shaded}

\begin{Shaded}
\begin{Highlighting}[]
\NormalTok{out<-ej4.sim}\OperatorTok{$}\NormalTok{sims.list}
\CommentTok{#Analisis de betas}
\NormalTok{betas<-}\ControlFlowTok{function}\NormalTok{(p,pos)\{}

\NormalTok{b<-out}\OperatorTok{$}\NormalTok{beta[,pos]}
\KeywordTok{par}\NormalTok{(}\DataTypeTok{mfrow=}\KeywordTok{c}\NormalTok{(}\DecValTok{2}\NormalTok{,}\DecValTok{2}\NormalTok{))}
\KeywordTok{plot}\NormalTok{(b,}\DataTypeTok{type=}\StringTok{"l"}\NormalTok{,}\DataTypeTok{xlab =}\NormalTok{ p)}
\KeywordTok{plot}\NormalTok{(}\KeywordTok{cumsum}\NormalTok{(b)}\OperatorTok{/}\NormalTok{(}\DecValTok{1}\OperatorTok{:}\KeywordTok{length}\NormalTok{(b)),}\DataTypeTok{type=}\StringTok{"l"}\NormalTok{,}\DataTypeTok{xlab =}\NormalTok{ p)}
\KeywordTok{hist}\NormalTok{(b,}\DataTypeTok{freq=}\OtherTok{FALSE}\NormalTok{, }\DataTypeTok{main =}\NormalTok{ p)}
\KeywordTok{acf}\NormalTok{(b)}
\NormalTok{\}}

\KeywordTok{betas}\NormalTok{(}\StringTok{"b0"}\NormalTok{,}\DecValTok{1}\NormalTok{)}
\end{Highlighting}
\end{Shaded}

\includegraphics{tarea4_files/figure-latex/unnamed-chunk-5-1.pdf}

\begin{Shaded}
\begin{Highlighting}[]
\KeywordTok{betas}\NormalTok{(}\StringTok{"b1"}\NormalTok{,}\DecValTok{2}\NormalTok{)}
\end{Highlighting}
\end{Shaded}

\includegraphics{tarea4_files/figure-latex/unnamed-chunk-5-2.pdf}

\begin{Shaded}
\begin{Highlighting}[]
\KeywordTok{betas}\NormalTok{(}\StringTok{"b2"}\NormalTok{,}\DecValTok{3}\NormalTok{)}
\end{Highlighting}
\end{Shaded}

\includegraphics{tarea4_files/figure-latex/unnamed-chunk-5-3.pdf}

\begin{Shaded}
\begin{Highlighting}[]
\KeywordTok{betas}\NormalTok{(}\StringTok{"b3"}\NormalTok{,}\DecValTok{4}\NormalTok{)}
\end{Highlighting}
\end{Shaded}

\includegraphics{tarea4_files/figure-latex/unnamed-chunk-5-4.pdf}

\begin{Shaded}
\begin{Highlighting}[]
\NormalTok{z<-out}\OperatorTok{$}\NormalTok{beta}
\KeywordTok{par}\NormalTok{(}\DataTypeTok{mfrow=}\KeywordTok{c}\NormalTok{(}\DecValTok{1}\NormalTok{,}\DecValTok{1}\NormalTok{))}
\KeywordTok{plot}\NormalTok{(z)}
\end{Highlighting}
\end{Shaded}

\includegraphics{tarea4_files/figure-latex/unnamed-chunk-6-1.pdf}

\begin{Shaded}
\begin{Highlighting}[]
\KeywordTok{pairs}\NormalTok{(z)}
\end{Highlighting}
\end{Shaded}

\includegraphics{tarea4_files/figure-latex/unnamed-chunk-6-2.pdf}

\begin{Shaded}
\begin{Highlighting}[]
\CommentTok{#Resumen (estimadores)}
\CommentTok{#OpenBUGS}
\NormalTok{out.sum<-ej4.sim}\OperatorTok{$}\NormalTok{summary}
\CommentTok{#Tabla resumen}
\NormalTok{out.sum.t<-out.sum[}\KeywordTok{grep}\NormalTok{(}\StringTok{"beta"}\NormalTok{,}\KeywordTok{rownames}\NormalTok{(out.sum)),}\KeywordTok{c}\NormalTok{(}\DecValTok{1}\NormalTok{,}\DecValTok{3}\NormalTok{,}\DecValTok{7}\NormalTok{)]}
\NormalTok{out.sum.t<-}\KeywordTok{cbind}\NormalTok{(out.sum.t,}\KeywordTok{apply}\NormalTok{(out}\OperatorTok{$}\NormalTok{beta,}\DecValTok{2}\NormalTok{,prob))}
\KeywordTok{dimnames}\NormalTok{(out.sum.t)[[}\DecValTok{2}\NormalTok{]][}\DecValTok{4}\NormalTok{]<-}\StringTok{"prob"}
\KeywordTok{print}\NormalTok{(out.sum.t)}
\end{Highlighting}
\end{Shaded}

\begin{verbatim}
##              mean       2.5%     97.5%         prob
## beta[1] 17.877896 13.5400000 21.725249 0.0000000000
## beta[2]  1.135887  0.4633950  1.787525 0.0002222222
## beta[3]  0.323726  0.2528475  0.396600 0.0000000000
## beta[4]  1.244461  0.5955475  1.849525 0.0006666667
\end{verbatim}

Al igual que lo muestran los histogramas de cada \(\beta\) vemos que
cada una es significativamente distinta de cero ya que su intervalo de
confianza al 95\% no contiene al cero.

\begin{Shaded}
\begin{Highlighting}[]
\CommentTok{#Predictions}
\NormalTok{out.yf<-out.sum[}\KeywordTok{grep}\NormalTok{(}\StringTok{"yf"}\NormalTok{,}\KeywordTok{rownames}\NormalTok{(out.sum)),]}
\NormalTok{or<-}\KeywordTok{order}\NormalTok{(sal}\OperatorTok{$}\NormalTok{X1)}
\NormalTok{ymin<-}\KeywordTok{min}\NormalTok{(sal}\OperatorTok{$}\NormalTok{Y,out.yf[,}\KeywordTok{c}\NormalTok{(}\DecValTok{1}\NormalTok{,}\DecValTok{3}\NormalTok{,}\DecValTok{7}\NormalTok{)])}
\NormalTok{ymax<-}\KeywordTok{max}\NormalTok{(sal}\OperatorTok{$}\NormalTok{Y,out.yf[,}\KeywordTok{c}\NormalTok{(}\DecValTok{1}\NormalTok{,}\DecValTok{3}\NormalTok{,}\DecValTok{7}\NormalTok{)])}
\KeywordTok{par}\NormalTok{(}\DataTypeTok{mfrow=}\KeywordTok{c}\NormalTok{(}\DecValTok{1}\NormalTok{,}\DecValTok{1}\NormalTok{))}
\KeywordTok{plot}\NormalTok{(sal}\OperatorTok{$}\NormalTok{X1,sal}\OperatorTok{$}\NormalTok{Y,}\DataTypeTok{ylim=}\KeywordTok{c}\NormalTok{(ymin,ymax))}
\KeywordTok{lines}\NormalTok{(sal}\OperatorTok{$}\NormalTok{X1[or],out.yf[or,}\DecValTok{1}\NormalTok{],}\DataTypeTok{lwd=}\DecValTok{2}\NormalTok{,}\DataTypeTok{col=}\DecValTok{2}\NormalTok{)}
\KeywordTok{lines}\NormalTok{(sal}\OperatorTok{$}\NormalTok{X1[or],out.yf[or,}\DecValTok{3}\NormalTok{],}\DataTypeTok{lty=}\DecValTok{2}\NormalTok{,}\DataTypeTok{col=}\DecValTok{2}\NormalTok{)}
\KeywordTok{lines}\NormalTok{(sal}\OperatorTok{$}\NormalTok{X1[or],out.yf[or,}\DecValTok{7}\NormalTok{],}\DataTypeTok{lty=}\DecValTok{2}\NormalTok{,}\DataTypeTok{col=}\DecValTok{2}\NormalTok{)}
\end{Highlighting}
\end{Shaded}

\includegraphics{tarea4_files/figure-latex/unnamed-chunk-8-1.pdf}

\begin{Shaded}
\begin{Highlighting}[]
\KeywordTok{plot}\NormalTok{(out.yf[}\DecValTok{1}\OperatorTok{:}\DecValTok{24}\NormalTok{,}\DecValTok{1}\NormalTok{],sal}\OperatorTok{$}\NormalTok{Y,}\DataTypeTok{type =} \StringTok{"p"}\NormalTok{)}
\end{Highlighting}
\end{Shaded}

\includegraphics{tarea4_files/figure-latex/unnamed-chunk-8-2.pdf}

Gráficamente se muestra un buen ajuste del modelo.

Futuros\ldots{}

\begin{Shaded}
\begin{Highlighting}[]
\CommentTok{#Tabla resumen}
\NormalTok{out.sum.f<-out.sum[}\KeywordTok{grep}\NormalTok{(}\StringTok{"yf1"}\NormalTok{,}\KeywordTok{rownames}\NormalTok{(out.sum)),}\KeywordTok{c}\NormalTok{(}\DecValTok{1}\NormalTok{,}\DecValTok{3}\NormalTok{,}\DecValTok{7}\NormalTok{)]}
\NormalTok{out.sum.f<-}\KeywordTok{cbind}\NormalTok{(out.sum.f,}\KeywordTok{apply}\NormalTok{(out}\OperatorTok{$}\NormalTok{yf1,}\DecValTok{2}\NormalTok{,prob))}
\KeywordTok{dimnames}\NormalTok{(out.sum.f)[[}\DecValTok{2}\NormalTok{]][}\DecValTok{4}\NormalTok{]<-}\StringTok{"prob"}
\KeywordTok{print}\NormalTok{(out.sum.f)}
\end{Highlighting}
\end{Shaded}

\begin{verbatim}
##            mean     2.5%    97.5% prob
## yf1[1] 33.95618 29.97899 38.02000    0
## yf1[2] 67.20652 56.01849 78.41000    0
## yf1[3] 34.11680 30.07475 38.06525    0
\end{verbatim}

Dic usando beta{[}j{]} \textasciitilde{} dnorm(0,0.001) mejor ajuste

\begin{Shaded}
\begin{Highlighting}[]
\CommentTok{#DIC}

\NormalTok{out.dic<-ej4.sim}\OperatorTok{$}\NormalTok{DIC}
\CommentTok{#out.dic<-ej4.sim$BUGSoutput$DIC}
\KeywordTok{print}\NormalTok{(out.dic)}
\end{Highlighting}
\end{Shaded}

\begin{verbatim}
## [1] 101.4
\end{verbatim}

\(R^2\)

\begin{Shaded}
\begin{Highlighting}[]
\NormalTok{R2<-(}\KeywordTok{cor}\NormalTok{(sal}\OperatorTok{$}\NormalTok{Y,out.yf[}\DecValTok{1}\OperatorTok{:}\DecValTok{24}\NormalTok{,}\DecValTok{1}\NormalTok{]))}\OperatorTok{^}\DecValTok{2}
\KeywordTok{print}\NormalTok{(R2)}
\end{Highlighting}
\end{Shaded}

\begin{verbatim}
## [1] 0.9107619
\end{verbatim}


\end{document}
